\documentclass[12pt]{article}

\usepackage[a4paper,margin=1in]{geometry}
\usepackage{setspace}
\usepackage{lmodern}
\usepackage[english]{babel}
\usepackage{mathtools,amsthm,amssymb,amsmath}
% \numberwithin{equation}{section}
\usepackage{booktabs} % For better-looking tables
\usepackage{dcolumn}  % For aligning decimal points in tables
\usepackage{hyperref}
\usepackage{ marvosym }
\usepackage{eurosym}
\usepackage{bm}
\usepackage{graphicx}
\usepackage[longnamesfirst]{natbib}
\usepackage{float}
\usepackage{longtable}
\usepackage{caption}
\usepackage{listings}
\usepackage{xcolor}
\usepackage[numbib]{tocbibind}
\usepackage{adjustbox}
\usepackage{varwidth}
\usepackage{algpseudocodex}
\usepackage{algorithm}
\usepackage{subcaption}

\bibliographystyle{rug-econometrics}
\settocbibname{References}

\lstdefinelanguage{SQL}{
	keywords={SELECT, FROM, WHERE, INSERT, INTO, UPDATE, DELETE, CREATE, TABLE, ALTER, DROP, JOIN, ON, AND, OR, NOT, NULL, PRIMARY, KEY},
	keywordstyle=\color{blue}\bfseries,
	comment=[l]{--},
	commentstyle=\color{gray}\ttfamily,
	stringstyle=\color{red},
	morestring=[b]',
	morestring=[b]"
}

\lstdefinelanguage{Python}{
	keywords={def, return, if, elif, else, while, for, in, try, except, finally, with, as, import, from, class, pass, break, continue, and, or, not, is, None, True, False, lambda, yield, global, nonlocal, assert, raise},
	keywordstyle=\color{blue}\bfseries,
	comment=[l]\#,
	commentstyle=\color{gray}\ttfamily,
	stringstyle=\color{red},
	morestring=[b]',
	morestring=[b]"
}

\lstset{
	basicstyle=\ttfamily,
	columns=fullflexible,
	language=SQL,
	frame=single,
	backgroundcolor=\color{gray!10}
}

% \setlength{\parindent}{0pt}     % No paragraph indentation
% \setlength{\parskip}{1em}       % Add vertical space between paragraphs

\newcommand{\N}[1]{\mathcal{N}\left(#1\right)}
\newcommand{\Z}{\mathbb{Z}}
\DeclareMathOperator*{\argmin}{arg\,min}
\DeclareMathOperator*{\argmax}{arg\,max}
\newcommand{\abs}[1]{\left\vert#1\right\vert}
\newcommand{\given}{\,\middle|\,}
\newcommand{\Bern}[1]{\mathrm{Bern}(#1)}
\newcommand{\Bin}[1]{\mathrm{Bin}(#1)}
\newcommand{\Exp}[1]{\mathrm{Exp}(#1)}
\newcommand{\FS}[1]{\mathrm{FS}(#1)}
\newcommand{\Geo}[1]{\mathrm{Geo}(#1)}
\newcommand{\Norm}[1]{\mathrm{Norm}(#1)}
\newcommand{\Pois}[1]{\mathrm{Pois}(#1)}
\newcommand{\Unif}[1]{\mathrm{Unif}(#1)}
\newcommand{\E}[1]{\,\mathsf{E}\left[#1\right]}
\newcommand{\EE}[2]{\,\mathsf{E}_{#1}\left[#2\right]}
\newcommand{\V}[1]{\,\mathsf{V}\left[#1\right]}
\newcommand{\cov}[1]{\,\mathsf{Cov}\left[#1\right]}
\renewcommand{\d}[1]{\,\textrm{d}#1}
\newcommand{\1}[1]{\,I_{#1}} % indicator
\renewcommand{\P}[1]{\,\mathbb{P}\left\{#1\right\}}
\renewcommand{\cos}[1]{\text{cos}\left[#1\right]}
\newcommand{\fat}[1]{\ThisStyle{\hstretch{1.2}{\ooalign{%
				\kern.46pt$\SavedStyle#1$\cr\kern.33pt$\SavedStyle#1$\cr%
				\kern.2pt$\SavedStyle#1$\cr$\SavedStyle#1$}}}}
\renewcommand{\ln}[1]{\,\mathrm{ln}\left[#1\right]}
\newcommand*{\B}[1]{\ifmmode\bm{#1}\else\textbf{#1}\fi}
