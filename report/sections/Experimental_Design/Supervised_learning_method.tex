\subsection{Supervised learning method}
In the second part of this research, a supervised learning model is trained to predict the TSP path length from neighborhood-level features and the number of
points \( j \). Specifically, a Random Forest regression model is employed.

The data set is constructed using the results from Algorithm \ref{alg:evaluation}. Each entry of $L$ (the list of TSP path lengths) is a separate observation
in this data set. For every observation, the corresponding area features and the number of locations $j$ are stored in the data set. Then, the following steps
are performed:
\begin{enumerate}
	\item Split the data into a training and test set, 80/20 split.
	\item Normalize the features such that they all have mean 0 and variance 1.
	\item Train a Random Forest regressor to predict TSP path length \( L \) using the number of points \( j \) and neighborhood-level features as input variables.
	\item Evaluate model performance on the test set using the metrics (for the test data):
	      \begin{itemize}
		      \item Mean Absolute Error (MAE)
		      \item Mean Squared Error (MSE)
		      \item Coefficient of determination \( R^2 \)
	      \end{itemize}
	      % \item Analyze feature importance to assess which neighborhood characteristics most influence the TSP length.
\end{enumerate}

The goal of this part is to determine how accurately TSP path length can be predicted based on readily available spatial and demographic characteristics of
neighborhoods.
% and to identify which features are most influential.
