\subsection{Evaluation of BHH formula}
The following algorithm is used for the first part of the research.
Let \( L = \{L_i\} \) and \( \hat{L} = \{\hat{L}_i\} \) be the lists of observed and predicted TSP tour lengths,
respectively, aggregated over all sample sizes \( j \) and repetitions \( k \) for neighborhood \( n \).
Let \( |L| \) denote the total number of evaluated instances per neighborhood.
\begin{algorithm}[H]
	\caption{Procedure for evaluating the predictive accuracy of Equation~\ref{eq:beardwood}}
	\label{alg:evaluation}
	\begin{algorithmic}[1]
		\For{each neighborhood \( n \in N \)}
		\State Extract features of this neighborhood and save to a file for later use
		\State Construct a graph of the road network and visualize on a map
		\State Initialize empty lists for \( L \) and \( \hat{L} \)
		\For{each sample size \( j \in \{20, 22, \ldots, 86, 88\} \)}
		\For{each repetition \( k \in \{0, 1, \ldots, 10\} \)}
		\State Randomly sample a subset \( TSP_{j,n} \subset X_n \) of size \( j \) uniformly
		\State Solve the TSP over \( TSP_{j,n} \) to compute \( L_{j,n}^{(k)} \)
		\State Solve Equation \ref{eq:beardwood} for $\beta$ using these solutions \(L_{j,n}^{(k)}\)
		\State Compute the predicted length \( \hat{L}_{j,n}^{(k)} \) using Equation~\ref{eq:beardwood} with the estimated $\beta$
		\State Append \( L_{j,n}^{(k)} \) to list \( L \), and \( \hat{L}_{j,n}^{(k)} \) to list \( \hat{L} \)
		\EndFor
		\EndFor
		\State Compute the Mean Absolute Error (MAE) for neighborhood \( n \):
		\[
			\text{MAE}_n = \frac{1}{|L|} \sum_{i=1}^{|L|} \left| L_i - \hat{L}_i \right|
		\]
		\State Compute the Mean Absolute Percentage Error (MAPE) for neighborhood \( n \):
		\[
			\text{MAPE}_n = \frac{100\%}{|L|} \sum_{i=1}^{|L|} \left| \frac{L_i - \hat{L}_i}{L_i} \right|
		\]
		\State Visualize the results in graph form, a scatter plot of TSP length against number of locations, with
		a fitted equation \ref{eq:beardwood} and a histogram of the errors $\epsilon_i=L_i-\hat{L}_i$.
		\State Save the results for $L$ to a file for later use.
		\EndFor
	\end{algorithmic}
\end{algorithm}
