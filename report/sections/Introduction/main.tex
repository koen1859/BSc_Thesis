\section{Introduction}
The Traveling Salesman Problem (TSP) is an important problem in operations research.
It is particularly relevant for last-mile carriers and other logistics companies where efficient
routing directly impacts cost, time and service quality. Since the number of parcels worldwide has
increased between 2013 and 2022 and is expected to keep increasing \citep{statista}, the need for
fast, scalable route planning methods becomes ever more pressing.

The TSP is an NP-hard problem, it is computationally intensive to find the exact solution for
large instances. In many real-world scenarios, the exact optimal routes may not be needed, but
instead a rough, reliable estimate of the optimal route length. For instance, consider a postal delivery company.
This firm may need to assign a certain amount of deliveries or a certain area to each postman.
Reliable estimates for the route length can provide valuable information for making such decisions.

Efficient approximation methods provide a solution for such practical applications where exact
solutions are too computationally intensive to conduct or not feasible due to insufficient data.
These methods aim at approximating the expected optimal total travel time or distance, while using
minimal data and computational effort.

There is extensive research on such approximation methods and how they perform in the Euclidean
plane.
Consider $n$ uniformly drawn locations from some area in $\mathbb{R}^2$ with area $A$.
\citet{beardwood1959shortest} find the relation:
\begin{align}
	\frac{L}{\sqrt n} \to \beta \sqrt{A}, \quad \text{as } n \to \infty
	\label{eq:beardwood}
\end{align}
as an estimation for the length of the shortest TSP path measured by Euclidean distance through
these random locations, where $\beta$ is some proportionality constant. 

The BHH formula is a neat result, the formula makes intuitive sense. However, in contrast to the 
Euclidean plane, based on different features of the area, TSP path lengths in this area might 
behave differently. For example, when there are clusters of locations with empty space in between 
them, or natural boundaries such as canals could have impact on TSP path length in an area. The 
BHH formula might not be able to capture such complex relations, and fail to provide accurate results.
The value for $\beta$ might differ vastly across different types of areas, as a result of varying geographic
features.
This would result in having to use a different value for $\beta$ for every different area, making it less 
viable to use this formula for delivery services, especially they do not know the true value of $\beta$ for the areas they serve.
A potential solution to this might be to use supervised learning methods to predict the length of the shortest TSP path, 
aimed at capturing more complex relations between area features and the TSP path length in this area.
This would result in one central formula to predict TSP path length accurately, removing the need to calculate
separate parameters for each area.

This research investigates how well Equation \ref{eq:beardwood} performs, predicting TSP path length on real road networks. 
Using OpenStreetMap \citep{openstreetmap} data, TSP instances are simulated in a wide variety of different urban areas
in the Netherlands, then solve these for the actual shortest paths using the Lin–Kernighan heuristic \citep{lin1973effective}.
Then, the $\beta$ from equation \ref{eq:beardwood} is estimated and the performance of this formula is analyzed. 
Additionally, the results for $\beta$ and the performance across the selected areas is compared. In an attempt to capture
more complex relations between area features and TSP path length, a supervised learning method is used as an alternative 
model to predict TSP paths.

The core contributions of this research are:
\begin{enumerate}
	\item An analysis of the BHH formula under the following conditions:
	      \begin{enumerate}
		      \item The locations are drawn from the set of postcodes in the area in question, a relaxation of the uniform distribution of visited
		            locations. In reality, delivery locations for companies are not uniformly distributed across the delivery area, but they are clustered
		            in certain parts of the area, for example in high rise buildings;
		      \item Applied to realistically sized (in the sense that a delivery person could serve this area in a single workday) real-world parts of
		            cities and villages in the Netherlands;
	      \end{enumerate}
	\item A supervised learning analysis to investigate how well the optimal TSP path length can be
	      predicted, based on features of the area the path is in, including road network density,
	      address density and other natural and man-made features of the area.
\end{enumerate}
The analysis can easily be extended to any type of area in any part of the world, one would only
have to download the OpenStreetMap \citep{openstreetmap} for another part of the world and add the names of the areas
to apply it to. The source code of this project is available on
\href{https://github.com/koen1859/BSc_Thesis}{GitHub}.

In section 2 the context and previous research in this field is provided.
In section 3 a formal problem definition is stated. 
Section 4 contains a description of the data that is used, and explains how it is processed.
In section 5 the design of the experiment is documented, section 6 displays the results.
In section 8 the research is concluded.
