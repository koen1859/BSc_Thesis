\section{Introduction}
The Traveling Salesman Problem (TSP) is an important problem in operations research.
It is particularly relevant for last-mile carriers and other logistics companies where efficient
routing directly impacts cost, time and service quality. Since the number of parcels worldwide has
increased between 2013 and 2022 and is expected to keep increasing \citep{statista}, the need for
fast, scalable route planning methods becomes ever more pressing.

The TSP is an NP-hard problem, it is computationally intensive to find the exact solution for
large instances. In many real-world scenarios, the exact optimal routes may not be needed, but
instead a rough, reliable estimate of the optimal route length. For instance, consider a postal delivery company.
This firm may need to assign a certain amount of deliveries or a certain area to each postman.
Reliable estimates for the route length can provide valuable information for making such decisions.

Efficient approximation methods provide a solution for such practical applications where exact
solutions are too computationally intensive to conduct or not feasible due to insufficient data.
These methods aim at approximating the expected optimal total travel time or distance, while using
minimal data and computational effort.

There is extensive research on such approximation methods and how they perform in the Euclidean
plane.
Consider $n$ uniformly drawn locations from some area in $\mathbb{R}^2$ with area $A$.
\citet{beardwood1959shortest} prove the relation:
\begin{align}
	L \to \beta \sqrt{nA}, \quad \text{as } n \to \infty
	\label{eq:beardwood}
\end{align}
as an estimation for the length of the shortest TSP path measured by Euclidean distance through
these random locations, where $\beta$ is some proportionality constant. This formula is a very
elegant result, and it requires very little data. However, its assumptions,
uniform random locations and euclidean space differ from real-world applications, which are defined
by complex geographic features, such as road networks.

This research investigates how well this approximation method performs when considering real road
networks. Using OpenStreetMap \citep{openstreetmap} data, TSP instances are simulated in a wide variety of different urban areas
in the Netherlands, then solve these for the actual shortest paths using the Lin–Kernighan heuristic
\citep{lin1973effective}.
Then, the $\beta$ from equation \ref{eq:beardwood} is estimated and the performance of this
formula is analyzed. Additionally, the results for $\beta$ and the performance across the selected
areas is compared. 

The core contribution of this research is the performance of the Beardwood formula is analyzed when: 
\begin{enumerate}
  \item relaxing the assumption of uniformly drawn locations. In this research,
	the locations are drawn from the set of postcodes in the area in question.
  \item applied to realistically sized real-world parts of cities and villages in the Netherlands.
\end{enumerate}
The analysis can easily be extended to any type of area in any part of the world, one would only
have to download the OpenStreetMap \citep{openstreetmap} for another part of the world and add the names of the areas
to apply it to. The source code of this project is available on 
\href{https://github.com/koen1859/Bsc_Thesis}{GitHub}.

In section 2 a deep dive in the context and previous research in this field is provided.
In section 3 the experimental design is documented.
