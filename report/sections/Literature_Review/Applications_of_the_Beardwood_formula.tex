\subsection{Applications of the BHH formula}
This research concerns the performance of formula \ref{eq:beardwood} for reasonable amounts of
locations a delivery person can visit in a workday, say $10\leq n\leq90$.
\citet{lei2015dynamic} estimates the values of $\beta$ for a selection of values for $n$.
In their research, the points were generated uniformly and the $L_2$ distance metric was used.
Table \ref{tab:beta-values} lists the results.
\begin{table}[H]
	\centering
	\caption{Empirical estimates of $\beta$ as a function of $n$, $20 \leq n \leq 90$\\
		\citep{lei2015dynamic}}
	\label{tab:beta-values}
	\begin{tabular}{cc}
		\toprule
		$n$ & $\beta(n)$ \\
		\midrule
		20  & 0.8584265  \\
		30  & 0.8269698  \\
		40  & 0.8129900  \\
		50  & 0.7994125  \\
		60  & 0.7908632  \\
		70  & 0.7817751  \\
		80  & 0.7775367  \\
		90  & 0.7773827  \\
		\bottomrule
	\end{tabular}
\end{table}
\citet{figliozzi2008planning} is the first research to apply approximation formulas to real-world
instances of TSPs (and VRPs (Vehicle Routing Problems)). An extension of formula
\ref{eq:beardwood} that works for VRPs is assessed in a real-world setting. It is found that this
model has an $R^2$ of 0.99 and MAPE (Mean Absolute Prediction Error) of 4.2\%. This prediction error
is slightly higher than when it is applied to a setting where Euclidean distances are considered (3.0\%),
but the formula still performs well \citep{figliozzi2008planning}.

\citet{merchan2019empirical} use circuity factors to measure the relative detour incurred for
traveling in a road network, compared to the Euclidean distance. This circuity factor is defined
as, where $p$ and $q$ are locations:
\begin{align}
	c = \frac{d_{c}(p,q)}{d_{L_{2}}(p,q)}
	\label{eq:circuity}
\end{align}
By construction, $c$ is greater or equal to 1, a value closer to 1 indicates a more efficient network. Then, $\beta_c$
is estimated by $\beta_c=c\beta$. This value $c$, is estimated for three different areas in
São Paulo, for which the results are listed in table \ref{tab:beta-merchan}. These values indicate
real travel distances are on average 2.76 times longer in area 1 compared to the $L_2$ metric.
These values were obtained by uniformly generating $n$ locations (for $n$ ranging from 3 to 250),
computing near-optimal tour lengths under the Euclidean metric, and solving for $\beta$, then
scaling by the empirical circuity factor.
\begin{table}[htbp]
	\centering
	\caption{Estimates of the circuity factor $c$ and its corresponding $\beta_c$ \citep{merchan2019empirical}}
	\label{tab:beta-merchan}
	\begin{tabular}{lccc}
		\toprule
		          & Area 1 & Area 2 & Area 3 \\
		\midrule
		$c$       & 2.76   & 2.34   & 1.82   \\
		$\beta_c$ & 2.48   & 2.10   & 1.64   \\
		\bottomrule
	\end{tabular}
\end{table}
It is important to note, however, that the assumptions in this study may limit the generality of
the findings. In particular, the use of uniformly distributed locations does not accurately reflect
the spatial distribution of delivery points in real urban environments, where locations tend to
cluster in residential, commercial, or industrial zones. Additionally, within small urban areas,
high-rise buildings and single-family homes may coexist in the same neighborhoods, further
challenging the assumption of uniformly distributed delivery points.
Furthermore, the circuity factor $c$ can
vary significantly within a single city, depending on local street patterns, infrastructure, and
topography. These variations suggest that a fixed circuity factor may oversimplify the complexity
of real-world delivery contexts, especially when applied to smaller sub-regions or neighborhoods.
